\documentclass[12pt]{article} 

\usepackage{titling}
\usepackage{extsizes} 

\pretitle{\vspace*{2in}\begin{center}\LARGE\bfseries} % Adjust vertical space before title
\posttitle{\par\end{center}\vskip 0.5em}
\preauthor{\begin{center}\large}
\postauthor{\end{center}}

\newcommand{\subtitle}[1]{%
    \posttitle{%
        \par\end{center}
        \begin{center}\Large#1\end{center}
        \vskip 0.5em}%
}
\title{Projeto de Compiladores 2024/25}
\subtitle{Compilador para a linguagem deiGO}
\author{Trabalho realizado por:\\
Miguel Castela uc2022212972 \\
Nuno Batista uc2022212972}
\date{}
\begin{document}

\maketitle

\newpage

\section{Introdução}
Este relatório descreve o desenvolvimento de um compilador para a linguagem deiGO, realizado no âmbito do projeto da disciplina de Compiladores do ano letivo 2024/25. O objetivo deste projeto é aplicar os conhecimentos adquiridos ao longo do curso na construção de um compilador funcional.

\section{Gramática}
A linguagem deiGO é uma linguagem de programação criada especificamente para este projeto. Ela possui características que permitem a implementação de conceitos fundamentais de compiladores, como análise léxica, análise sintática, geração de código intermediário e otimização.

\section{Algoritmos e Estruturas de Dados}
O compilador desenvolvido é composto por várias fases, cada uma responsável por uma parte específica do processo de compilação. As principais fases são:

\begin{itemize}
    \item Análise Léxica
    \item Análise Sintática
    \item Análise Semântica
    \item Geração de Código Intermediário
    \item Otimização
    \item Geração de Código Final
\end{itemize}

\section{Geração de Código}


\end{document}